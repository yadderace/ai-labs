\documentclass{article}

% Language setting
\usepackage[english]{babel}

% Set page size and margins
% Replace `letterpaper' with `a4paper' for UK/EU standard size
\usepackage[letterpaper,top=2cm,bottom=2cm,left=3cm,right=3cm,marginparwidth=1.75cm]{geometry}

% Useful packages
\usepackage{amsmath}
\usepackage{graphicx}
\usepackage[colorlinks=true, allcolors=blue]{hyperref}
\usepackage{titling} % For subtitle support

\title{Network Science}
\author{Yadder Aceituno}

\begin{document}

\maketitle

The bridges of Königsberg:

Can I have a path that crosses each bridge exactly once?
The answer is no.

Computational Social Science:

Interdisciplinary field that uses computational methods to study social systems.

- How to connect nodes
- Nodes have attributes
- Quantity attached to an edge is a weight

If chatgpt can creates synthetic data, how to measure how good is.

A network is a graph.
Network science is data driven
Networks are collections of vertices joined by edges.
Vertex are nodes.
Edges are links. They can be referred as ties, bonds, connections, interactions.

A graph $$ G = (V, E) $$ where $$ V $$ is the set of vertices and $$ E $$ is the set of edges.
Each edge $$ e \in E $$ is a tuple $$ (u, v) $$ where $$ u, v \in V $$.
$$ u $$ and $$ v $$ are the endpoints of the edge and are said to be adjacent.

The max number of edges in a graph is $$ |E| = \binom{|V|}{2} $$.

Lecture 2

How do we represent a graph?

- Visual
- Adjacency matrix
- Edge List (This is most common in projects)

Types of networks:

- Undirected, simple
- Directed: also called
- Network with self-edges
- Multiedges
- Weighted graphs
- Hypergraphs
- Trees
- Bipartite networks
- Planar networks

There are two types of transformation from Undirected to Directed:

- Co-citation networks
- Bibliographic coupling networks

Degree of a node: number of edges incident to it.

Regular graph: all nodes have the same degree.

Vertex degree for directed graph


Forbidden Triad


------- Lecture 5

What do we know so far?

- A netfork is defined by a collection of nodes and edges
- There are different types of networks
- There are different network models
    - Erdős–Rényi Random graph
    - Barabási-Albert model
    - Watts-Strogatz model
- We have some network metrics for the nodes
    - Clustering Coefficient
    - Node Degree
- We have some network metrics for the network
    - Average Path Length
    - Average Shortest Path Length
    - Avg Clustering Coefficient
    - Density
    - Singleton
    - Degree Distribution
    - Diameter: Largest shortest path length
- How do we compare nodes?
    - Clustering Coefficient
    - Node Degree
    - Node Betweenness Centrality
    

Betweenness Centrality: 
Extent that a node lies on the shortest path between other nodes.
$$ X_i = \sum_{s \neq i \neq t} \frac{\sigma_{st}(i)}{\sigma_{st}} $$

Degree Centrality: 
The importance of a node is a function of the importance of its neighbors.

Kotz Centrality: 
$$ X_i = \sum_{j \in N(i)} X_j $$

Page Route


Closeness Centrality: 
$$ X_i = \frac{1}{\sum_{j \in N(i)} d_{ij}} $$


Similarity
    - Structural equivalence: nodes sharing same neighbords
    - Regultar equivalence: nodes sharing neighbors that are similar
    

\end{document}


