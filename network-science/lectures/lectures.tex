\documentclass{article}

% Language setting
\usepackage[english]{babel}

% Set page size and margins
% Replace `letterpaper' with `a4paper' for UK/EU standard size
\usepackage[letterpaper,top=2cm,bottom=2cm,left=3cm,right=3cm,marginparwidth=1.75cm]{geometry}

% Useful packages
\usepackage{amsmath}
\usepackage{graphicx}
\usepackage[colorlinks=true, allcolors=blue]{hyperref}
\usepackage{titling} % For subtitle support

\title{Network Science}
\author{Yadder Aceituno}

\begin{document}

\maketitle

The bridges of Königsberg:

Can I have a path that crosses each bridge exactly once?
The answer is no.

Computational Social Science:

Interdisciplinary field that uses computational methods to study social systems.

- How to connect nodes
- Nodes have attributes
- Quantity attached to an edge is a weight

If chatgpt can creates synthetic data, how to measure how good is.

A network is a graph.
Network science is data driven
Networks are collections of vertices joined by edges.
Vertex are nodes.
Edges are links. They can be referred as ties, bonds, connections, interactions.

A graph $$ G = (V, E) $$ where $$ V $$ is the set of vertices and $$ E $$ is the set of edges.
Each edge $$ e \in E $$ is a tuple $$ (u, v) $$ where $$ u, v \in V $$.
$$ u $$ and $$ v $$ are the endpoints of the edge and are said to be adjacent.

The max number of edges in a graph is $$ |E| = \binom{|V|}{2} $$.

Neighbor Set

How to represent a graph?

Multiedges

Self-edges

\end{document}


