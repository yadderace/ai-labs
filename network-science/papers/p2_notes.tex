\documentclass{article}

% Language setting
\usepackage[english]{babel}

% Set page size and margins
% Replace `letterpaper' with `a4paper' for UK/EU standard size
\usepackage[letterpaper,top=2cm,bottom=2cm,left=3cm,right=3cm,marginparwidth=1.75cm]{geometry}

% Useful packages
\usepackage{amsmath}
\usepackage{graphicx}
\usepackage[colorlinks=true, allcolors=blue]{hyperref}

\title{Socioeconomic reorganization of communication and mobilitynetworks in response to external shocks}
\author{Yadder Aceituno}

\begin{document}
\maketitle


\section{Summary Notes}

\subsection{Context}

External shocks, such as lockdowns, can lead to a reorganization of socioeconomic segregation 
patterns. The study was made in Sierra Leone. 

During lockdown, people physically stayed apart (more segregation in mobility), 
but socially they connected more widely across groups (less segregation in communication).

People tend to cluster with others who are similar to them in one or more traits, and this natural 
tendency creates patterns of segregation. The Socioeconomic Status (SES) is a key factor in this 
tendency.

People have different levels of wealth and social status, and these differences cause inequalities 
in many parts of life.

Segregation patterns are generally stable over time, but they can change in response to external shocks.
COVID-19 lockdowns are an example of such a shock. During lockdown, wealthier people adjusted their 
mobility patterns more easily (avoiding public transport, for example) than poorer people.

Because poorer people had fewer resources and less access to information, they were more exposed to the virus and 
suffered worse health outcomes, including higher death rates, during the pandemic. The question of the study was: 

\textbf{How did these abrupt behavioral changes and differences in adjustment capacities 
reorganize the social and mobility networks of people in the short terms?}

\subsection{Results}

Using the mobile phone call dataset, the authors studied from March 17, 2020 to April 17, 2020 (one month).

As a proxy for SES, first they assigned a RWI to each phone tower and with this they inferred the SES of each user.
With this location inference, they found that it was strong correlated with the actual population density distribution.
Demostrating that mobile phone users are representative of the population.

Then a social network $G_s$ was build where nodes are users and edges are time-varying communcation interactions between the users.

A mobility network $G_m$ was also built where nodes represent their homes and edges their visiting patterns.

Segregation patterns can be measured using the assortativity coefficient and can be clearly visualized using the 
assortativity matrix that shows the correlation between the SES of the nodes. The assortativity coefficient is:

$$
    \rho = \frac{\sum_{i,j} (x_i - \bar{x})(y_j - \bar{y})}{\sqrt{\sum_{i,j} (x_i - \bar{x})^2 \sum_{i,j} (y_j - \bar{y})^2 }}
$$

If $\rho = 1$ then the nodes are perfectly assortative, 
if $\rho = -1$ then the nodes are perfectly disassortative.

The lockdown and the curfew had radical effects on the socioeconomic segregation patterns. We can see in \ref{Figure 1}
that there's an assortative tendency in both mobility and social networks. Also, we can see that assortativity changed
during the lockdown and the curfew in different ways for each network. For the mobility network, assortativity increased
during the lockdown, but during the curfew went back to its mean value (slightly above). For the social network, assortativity 
decreased during the lockdown, but during the curfew went back to its mean value (slightly below). 

\begin{figure}[h]
    \centering
    \includegraphics[width=0.3\textwidth]{images/image1.png}
    \caption{Assortativity of mobility and social networks during the lockdown and the curfew}
    \label{Figure 1}
\end{figure}

This results suggests that people before the lockdown are  organized in highly segregated structures, it means that 
people with similar SES tend to interact more with each other. During the lockdown, the social network (communication like calls and texts)
became less segregated, but the mobility network (visits to different places) became more segregated.

The initial segregation and the way networks changed later are both explained by how Sierra Leone's society is 
structured, there's a big gap between rich and poor, and between city and countryside.

\begin{figure}[h]
    \centering
    \includegraphics[width=0.3\textwidth]{images/image2.png}
    \caption{Volume of calls/SMS and trips during the lockdown and the curfew}
    \label{Figure 2}
\end{figure}

In the paper, the authors wanted to explore why the effects on socioeconomic segregation patterns were different 
for the mobility and social networks. They assigned a class to each edge in both networks. A class could be WA (Western Area)
or OWA (Outer Western Area) or WA-OWA (Western Area and Outer Western Area). Then they calculated the volume of interactions 
between each class in both networks.

We can see in figure \ref{Figure 2} that the volume of interactions between classes in the mobility network fell
significantly during the lockdown and the curfew, which is expected. For the social network, the volume of interactions
also fell, but we can identify that interactions between WA and OWA increased were slightly above compared to the 
others.

To follow the changes around an individual $u$, the authors calculated the individual assortativity coefficient $r_u(t)$.
This quantifies the homogeneity of the SES of the neighbors of $u$ at time $t$. The results are showed in the next figure.

\begin{figure}[h]
    \centering
    \includegraphics[width=0.9\textwidth]{images/image3.png}
    \caption{Analysis of individual assortativity coefficient $r_u(t)$}
    \label{Figure 3}
\end{figure}

Analyzing the mobility network, the results show that poorest and richest peoeple were segregated during the lockdown and the curfew.
In Class 1 (poorest), the median in the reference perios (R1 and R2) was around 0.4, in the lockdown period the median
was around 1.3.
In Class 9 (richest), the median in the reference perios (R1 and R2) was around 0.4 as well, in the lockdown period the median
was around 0.6.
Analyzing the social network results, we can see that the poorest classes were the ones with more seggregation effects.

Analyzing the social network results, we can see that the poorest classes were the ones that were became less segregated.
Surprisingly, the richest classes were the ones that were became highly more segregated. In the chart we can see how the median
is shifted to the left for Class 1 and it's shifted to the right for Class 9.

\section{Questions}

\begin{itemize}
    \item From a technical perspective, what attributes can be used to determine SES?
    
    Places of residence, income, education level, job type, etc.
    
    \item This study was done in Sierra Leone, can we apply the same analysis to other countries?
    
    \item How did they assigned a tower to each user?
    
    \item How did they capture when a user moved to a different place, it was necessary that users make a call?
    
    \item Can we find the same results in other countries during lockdowns and curfews?
    
    \item Why there's a fall in the volume of social interactions (calls and texts), I was expecting an increase?
    
\end{itemize}



\end{document}
