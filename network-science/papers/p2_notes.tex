\documentclass{article}

% Language setting
\usepackage[english]{babel}

% Set page size and margins
% Replace `letterpaper' with `a4paper' for UK/EU standard size
\usepackage[letterpaper,top=2cm,bottom=2cm,left=3cm,right=3cm,marginparwidth=1.75cm]{geometry}

% Useful packages
\usepackage{amsmath}
\usepackage{graphicx}
\usepackage[colorlinks=true, allcolors=blue]{hyperref}

\title{Socioeconomic reorganization of communication and mobilitynetworks in response to external shocks}
\author{Yadder Aceituno}

\begin{document}
\maketitle


\section{Summary Notes}

\subsection{Context}

External shocks, such as lockdowns, can lead to a reorganization of socioeconomic segregation 
patterns. The study was made in Sierra Leone. 

During lockdown, people physically stayed apart (more segregation in mobility), 
but socially they connected more widely across groups (less segregation in communication).

People tend to cluster with others who are similar to them in one or more traits, and this natural 
tendency creates patterns of segregation. The Socioeconomic Status (SES) is a key factor in this 
tendency.

People have different levels of wealth and social status, and these differences cause inequalities 
in many parts of life.

Segregation patterns are generally stable over time, but they can change in response to external shocks.
COVID-19 lockdowns are an example of such a shock. During lockdown, wealthier people adjusted their 
mobility patterns more easily (avoiding public transport, for example) than poorer people.

Because poorer people had fewer resources and less access to information, they were more exposed to the virus and 
suffered worse health outcomes, including higher death rates, during the pandemic. The question of the study was: 

\textbf{How did these abrupt behavioral changes and differences in adjustment capacities 
reorganize the social and mobility networks of people in the short terms?}

\subsection{Results}

Using the mobile phone call dataset, the authors studied from March 17, 2020 to April 17, 2020 (one month).

As a proxy for SES, first they assigned a RWI to each phone tower and with this they inferred the SES of each user.
With this location inference, they found that it was strong correlated with the actual population density distribution.
Demostrating that mobile phone users are representative of the population.

Then a social network $G_s$ was build where nodes are users and edges are time-varying communcation interactions between the users.

A mobility network $G_m$ was also built where nodes represent their homes and edges their visiting patterns.

Segregation patterns can be measured using the assortativity coefficient and can be clearly visualized using the 
assortativity matrix that shows the correlation between the SES of the nodes. The assortativity coefficient is:

$$
    \rho = \frac{\sum_{i,j} (x_i - \bar{x})(y_j - \bar{y})}{\sqrt{\sum_{i,j} (x_i - \bar{x})^2 \sum_{i,j} (y_j - \bar{y})^2 }}
$$

If $\rho = 1$ then the nodes are perfectly assortative, 
if $\rho = -1$ then the nodes are perfectly disassortative.

The lockdown had radial effects on the social networks

\section{Questions}

\begin{itemize}
    \item From a technical perspective, what attributes can be used to determine SES?
    
    Places of residence, income, education level, job type, etc.
    
    \item This study was done in Sierra Leone, can we apply the same analysis to other countries?
    
    \item How did they assigned a tower to each user?
    
    \item How did they capture when a user moved to a different place, it was necessary that users make a call?
\end{itemize}



\end{document}
