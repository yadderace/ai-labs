\documentclass{article}

% Language setting
\usepackage[english]{babel}

% Set page size and margins
% Replace `letterpaper' with `a4paper' for UK/EU standard size
\usepackage[letterpaper,top=2cm,bottom=2cm,left=3cm,right=3cm,marginparwidth=1.75cm]{geometry}

% Useful packages
\usepackage{amsmath}
\usepackage{graphicx}
\usepackage[colorlinks=true, allcolors=blue]{hyperref}

\title{The strength of long-range ties in population-scale social networks}
\author{Yadder Aceituno}

\begin{document}
\maketitle


\section{Introduction}

It was assumed that long-range connections are weak. However, researches have shown 
that they are almost as strong as ties embedded in a small circle of friends.

\textbf{Source}: https://www.science.org/doi/10.1126/science.aau9735

\section{Notes}

Information that one acquires from within a small circle of friends is more likely to be the same 
as the information that one acquires from acquaintances is distant regions of the network.

It has been difficult to prove that  because of the difficulty of obtaining data for population-scale social networks.
There are some studies that proves that there's a six degrees of separation in large social networks,
the strength has never been measured.

Now it this paper, data from phone networks, twitter confirmed that social ties tend to be weaker 
(lower call volume and and fewer text message exchanges). But the data also shows that above 
range four the ties are strong. So it was observed that tie strenght was increasing with distance.

But there could be some problems at the moment of meeasuring this: Strong embedded ties are wrong measured 
if there's missing data in common neighbors. But it was tested that missing data is not a problem at all because
each node has multiple neighbors and it would require a large amount of missing data to affect the results.

Some summary of the research:
First, topic modeling suggested that most of the network wormholes involve religious and cultural topics.
Second, temporal analysis suggested that network wormholes are more likely to be interpersonal rather
than instrumental or work-related. Specially, the activity of the wormholes was in non-work hours.
Finally, the strength of long-range ties was not a byproduct of physical distance.

Some possible explanations: 
- People on the edges of a social network, who have few connections, often end up forming long-range ties. 
This happens because they have limited time or attention and must choose between having a few close friends 
or many weak acquaintances. Since they have fewer friends overall, they also have fewer mutual friends — making 
their connections reach farther across the network.

- Weak ties are more likely to break over time. The stronger ones remain despite the distance.

- Different types of connections remain separated. 

\end{document}
