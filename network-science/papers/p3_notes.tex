\documentclass{article}

% Language setting
\usepackage[english]{babel}

% Set page size and margins
% Replace `letterpaper' with `a4paper' for UK/EU standard size
\usepackage[letterpaper,top=2cm,bottom=2cm,left=3cm,right=3cm,marginparwidth=1.75cm]{geometry}

% Useful packages
\usepackage{amsmath}
\usepackage{graphicx}
\usepackage[colorlinks=true, allcolors=blue]{hyperref}

\title{The structure and function of antagonistics ties in village social}
\author{Yadder Aceituno}

\begin{document}
\maketitle


\section{Summary Notes}

\subsection{Context}

The overall structure of a social network is determined not only by its positive ties butalso by the negative ties within it.

Positive interactions, such as cooperation andmutualism, are clearly critical for success, 
but the potential benefits (if any) of negativerelationships are less clear.

The results reveal asimilar probability of negative ties occurring within communities
as compared to between them.

People who have more negative outward ties become "farther away" from their original friends over time.



\section{Questions}

\begin{itemize}
    \item From a technical perspective, what attributes can be used to determine SES?
    
    Places of residence, income, education level, job type, etc.
    
    \item This study was done in Sierra Leone, can we apply the same analysis to other countries?
    
    \item How did they assigned a tower to each user?
    
    \item How did they capture when a user moved to a different place, it was necessary that users make a call?
    
    \item Can we find the same results in other countries during lockdowns and curfews?
    
    \item Why there's a fall in the volume of social interactions (calls and texts), I was expecting an increase?
    
\end{itemize}

\section{Metrics}

- Geodesic distance


\end{document}
