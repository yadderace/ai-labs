\documentclass{LSkill}  % use the custom class

% Load extra packages you might need
\usepackage{amsmath}   % for math
\usepackage{graphicx}  % for images
\usepackage{hyperref}  % for clickable links
\usepackage{url}       % for proper URL formatting
\usepackage{tocloft}   % for customizing table of contents

% Add dot leaders to TOC
\renewcommand{\cftsecleader}{\cftdotfill{\cftdotsep}}  % for sections
\renewcommand{\cftsubsecleader}{\cftdotfill{\cftdotsep}}  % for subsections

\begin{document}

% Defining the title, author and date
\title{%
    \Huge\bfseries Negamax \\[0.5em]
    \large\mdseries Artificial Intelligence Algorithms
}
\author{Yadder Aceituno}
\date{\today}

% Creating the title page
\maketitle 

% Defining my abstract
\begin{abstract}
This is my abstract
\end{abstract}

% Add table of contents
\tableofcontents

\section{Introduction}
This paper aims to briefly and simply describe the negamax algorithm \footnote{\label{fn:negamax}The paper has been compiled using the following Wikipedia page: \url{https://en.wikipedia.org/wiki/Negamax}}, which is a variant of the min-max algorithm. 
Negamax provides a more simplified implementation of the min-max algorithm by using a single player perspective. 
With this simplification, the algorithm clarifies the principle that one player's gain is the other player's loss.

In the following sections is described the theoretical background of the algorithm and its relation with the min-max algorithm. Then, the implementation is described and some common optimizations are discussed. Finally, some applications in adversarial search are presented that make use of the negamax algorithm. 

\section{Theoretical Background}
Talking about the theoretical background

\section{Algorithm}
Talking about the algorithm

\section{Applications}
Talking about applications

\bibliographystyle{plain}   
\bibliography{references}   


\end{document}